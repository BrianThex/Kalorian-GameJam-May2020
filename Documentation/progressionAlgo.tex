\documentclass{scrartcl}
%\documentclass{article}
\usepackage{graphicx}
\usepackage{geometry}
\usepackage[ngerman]{babel}
\usepackage{blindtext}
\usepackage{subcaption}
\usepackage{amsfonts}
\usepackage{amssymb}
\usepackage{amsmath}
\usepackage{float}
\usepackage{stmaryrd}
\usepackage{scrpage2}
\usepackage{bbold}
\usepackage{tikz}
\usepackage{dsfont}

\usepackage{ulem}

\usepackage[T1]{fontenc}
\usepackage{selinput}
\SelectInputMappings{
  adieresis={ä},
  germandbls={ß}
}
\usepackage{babel}
\usepackage[markup=noncolor]{changes}

\usepackage{algorithm}
\usepackage{algorithmic}

\usepackage{longtable}

\geometry{top = 2cm, left = 2cm}

\title{Kalorian Game Jam Progression}
\subtitle{An Algorithm to distribute Upgrade Cards across Matches}

\author{Vincent Kreuziger (Zeel)}

\begin{document}
\maketitle
\tableofcontents
\section{Overview}
Both players receive cards throughout a match and upon ending it. Amount and type of those cards are based on their performance throughout the match. 
\section{The Algorithm}
T: Set of different towers\\
$u_1^t, \ldots , u_n^t: $ Set of Standard upgrade cards for tower $t\in T$
$v_1^t, \ldots , v_n^t: $ Set of Special upgrade cards for tower $t\in T$

Function evaluate(x) for $x\in\mathds{R}$ assigning each x a unique $u_i$ or $v_i$ so all $u,v \in U\cup V$ get assigned to one x\footnote{Not the other way round: Each potential x is mapped to a specific card.}. The function will give out poor cards if the return value is close to 0 and progressively better cards the greater the absolute return value is. 
\paragraph{Examples:}
\begin{align*}
evaluate(x) & = 1 \; for\;  0<x\leq 0.5\\
 & = 2 \; for\;  0.5<x\leq 1\\
 & = 3 \; for\;  1<x\leq 1.5\\
 & = 4 \; for\;  1.5<x\leq 2\\
 & = \vdots\\
 & = n \; for\;  3 < x\\
 & = -1 \; for\;  0\geq x < -0.5\\
 & = -2 \; for\;  0.5\geq x < -1\\
 & = -3 \; for\;  -1\geq x < -1.5\\
 & = -4 \; for\;  -1.5\geq x < -2\\
 & = \vdots\\
 & = -m \; for x < -3
\end{align*}
Changing those intervals changes the return values from the Gaussian distribution, meaning a card becomes more likely if its interval's absolute moves closer towards zero and vice versa. 
\label{example}

Function assign(i,t) assigning every integer within the range of evaluate(x)\footnote{Meaning: $-m-1, n \;\;\; \rightarrow$ see \ref{example}.}\\ \\ 

Number of cards received upon end of match: 
$$ n = (int(c_1/a_1)+max(int(a_2/e^{c_1}),a_3)+a_4$$

For each card received its variance is driven by the following function:
$$s = (c_1 + b_1)\cdot b_2$$


\begin{tabular}{l|l}
Var & Meaning\\
\hline
$a_{1-4}$ & constants\\
$b_{1,2}$ & constants\\
$c_1$ & waves survived by the players\\
$c_2$ & Cards left in the inventory\\
$a_1$ & every $a_1$ waves the players receive a card for reward\\
$a_2$ & weighting of remaining cards, high means more fresh cards while the player is running low\\
$a_3$ & covers the maximum so wasting all your cards doesn't yield 500 new draws\\
$a_4$ & minimum of cards the player draws\\
\hline
s	& Variance of quality, higher values mean for better upgrade cards\\
$b_1$ & Baseline applied to $b_2$ without considering the wave reached\\
& Guarantees a minimum of card variance even if the first wave isn't survived\\
$b_2$ & Weight of each wave, higher values make the waves more significant\\
\end{tabular}

We now roll a weighted die with T sides\footnote{Each side represents one type of tower.} n times. As a result we get the set $\{t_1,\ldots , t_n\}$ of towers that we receive cards for, a subset of all the towers available in game. Next we draw from a Gaussian Distribution N(0,s)\footnote{Mean 0, standard deviation s.}, which provides a sample $\{x_1,\ldots , x_n\}.$ As a result we get a set of n towers $\{n_1,\ldots ,t_n\}$ and a set of n gaussian distributed numbers $\{x_1, \ldots , x_n\}$. Our upgrades are now $assign(evaluate(x_i),t_i)$ for all i.

\section{Pseudocode}



\section{Notes}
\end{document}